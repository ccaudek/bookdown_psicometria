% !TEX encoding = UTF-8 
% !TeX spellcheck = Italian
%
% Corrado Caudek
% 2019/02/24

\chapter{Probabilità congiunta}
\label{chapter:distr_congiunta}


%\section*{Obiettivi di apprendimento}
%
%Lo studio di questo capitolo dovrebbe insegnare allo studente come:
%\begin{itemize}
%\item Spiegare il concetto di funzione di probabilità congiunta.
%\item Valutare se due variabili aleatorie sono indipendenti tramite l'uso delle funzioni di densità marginali e della densità congiunta.
%\end{itemize}
%
%
%\section*{Motivazione}

Finora abbiamo considerato unicamente le leggi di singole variabile aleatorie. 
Tuttavia, in psicologia e nella vita quotidiana, siamo spesso interessati a studiare problemi di probabilità legati al valore congiunto di due o più variabili aleatorie. 
Ad esempio, potremmo misurare il QI dei bambini e il loro peso alla nascita, o l'altezza e il peso delle giraffe, o il livello di inquinamento atmosferico e il tasso di malattie respiratorie nelle città, o il numero di amici di Facebook e l'età. 
Che relazione tra le variabili ci possiamo aspettare in ciascuno di questi esempi? 
Perché? 

Per capire la relazione che sussiste tra due variabili aleatorie è necessario calcolare gli indici di \emph{covarianza} e \emph{correlazione}. 
Per fare ciò è necessario utilizzare la funzione di probabilità congiunta.
L'obiettivo di questo capitolo è quello di chiarire cosa si intende per funzione di probabilità congiunta di due variabili casuali $X$ e $Y$.
Esamineremo in dettaglio il caso discreto e, proseguendo per analogia, forniremo solo dei brevi accenni al caso in cui $X$ e $Y$ sono variabili aleatorie continue.


\section{Caso discreto}

\subsection{Funzione di probabilità congiunta}

Dopo aver trattato delle distribuzioni di probabilità di una variabile aleatoria, che associa ad ogni evento elementare dello spazio campionario uno ed un solo numero reale, è naturale estendere questo concetto al caso di due o più dimensioni. 
Iniziamo a descrivere il caso discreto con un esempio. 
Consideriamo qui l'esperimento casuale corrispondente al lancio di tre monete equilibrate. 
Lo spazio campionario di tale esperimento casuale è
\[
\Omega = \{TTT, TTC, TCT, CTT, CCT, CTC, TCC, CCC\}.
\]
Dato che i tre lanci sono tra loro indipendenti, non c'è ragione di aspettarsi che uno degli otto risultati possibili dell'esperimento sia più probabile degli altri, dunque possiamo associare a ciascuno degli otto eventi elementari dello spazio campionario la stessa probabilità, ovvero 1/8.

Su tale spazio campionario consideriamo le variabili aleatorie $X \in \{0, 1, 2, 3\}$, che conta il numero delle teste nei tre lanci, e $Y \in \{0, 1\}$, che conta il numero delle teste al primo lancio. 
Indicando con T = `testa' e C = `croce', si ottiene dunque la situazione riportata nella tabella~\ref{tab:tre_monete_distr_cong_1}.
\begin{table}[h!]
\caption{Spazio campionario dell'esperimento consistente nel lancio di tre monete equilibrate su cui sono state definite le variabili aleatorie $X$ e $Y$.}
\label{tab:tre_monete_distr_cong_1}
\begin{center}
\begin{tabular}{cccc}
\toprule
$\omega$ & $X$ & $Y$ & $P(\omega)$ \\
\midrule
$\omega_1$ = TTT & 3 & 1 & 1/8 \\
$\omega_2$ = TTC & 2 & 1 & 1/8 \\
$\omega_3$ = TCT & 2 & 1 & 1/8 \\
$\omega_4$ = CTT & 2 & 0 & 1/8 \\
$\omega_5$ = CCT & 1 & 0 & 1/8 \\
$\omega_6$ = CTC & 1 & 0 & 1/8 \\
$\omega_7$ = TCC & 1 & 1 & 1/8 \\
$\omega_8$ = CCC & 0 & 0 & 1/8\\
\bottomrule
\end{tabular}
\end{center}
\label{default}
\end{table}%

Ci poniamo il problema di associare un livello di probabilità ad ogni coppia $(x, y)$ definita su $\Omega$. 
La coppia $(X = 0, Y = 0)$ si realizza in corrispondenza di un solo evento elementare, ovvero CCC; avrà dunque una probabilità pari a $Pr(X=0, Y=0) = Pr(CCC) = 1/8$.
Nel caso della coppia $(X = 1, Y = 0)$ ci sono due eventi elementari che danno luogo al risultato considerato, ovvero, CCT e CTC; la probabilità $Pr(X=1, Y=0)$ sarà dunque data dall'unione delle probabilità dei due eventi elementari corrispondenti, cioé $Pr(X=1, Y=0) = Pr(CCT \:\cup\: CTC) = 1/8 + 1/8 = 1/4$.
Riportiamo qui sotto i calcoli svolti per tutti i possibili valori di $X$ e $Y$.
\begin{align}
P(X = 0, Y = 0) &= P(\omega_8 = CCC) = 1/8; \notag\\
P(X = 1, Y = 0) &= P(\omega_5 = CCT) + P(\omega_6 = CTC) = 2/8; \notag\\
P(X = 1, Y = 1) &= P(\omega_7 = TCC) = 1/8; \notag\\
P(X = 2, Y = 0) &= P(\omega_4 = CTT) = 1/8; \notag\\
P(X = 2, Y = 1) &= P(\omega_3 = TCT) + P(\omega_2 = TTC) = 2/8; \notag\\
P(X = 3, Y = 1) &= P(\omega_1 = TTT) = 1/8; \notag
\end{align}
\noindent Le probabilità così trovate possono essere riportate nella tabella~\ref{tab:tre_monete_distr_cong}.
\begin{table}[h!]
\caption{Distribuzione di probabilità congiunta per i risultati dell'esperimento consistente nel lancio di tre monete equilibrate.}
\label{tab:tre_monete_distr_cong}
\begin{center}
\begin{tabular}{c|ccc}
\toprule
$x\textbackslash y$ & 0 & 1 \\
\midrule
0 & 1/8 & 0 \\
1 & 2/8 & 1/8 \\
2 & 1/8 & 2/8 \\
3 & 0   & 1/8 \\
\bottomrule
\end{tabular}
\end{center}
\label{default}
\end{table}%
La tabella~\ref{tab:tre_monete_distr_cong} ci fornisce il risultato che cercavamo, ovvero la distribuzione di probabilità congiunta delle variabili aleatorie $X$ = ``numero di realizzazioni con il risultato testa nei tre lanci'' e $Y$ = ``numero di realizzazioni con il risultato testa nel primo lancio'' per l'esperimento casuale considerato.

Una generica funzione di probabilità congiunta bivariata può essere rappresentata come indicato nella tabella~\ref{tab_prob_cong}.
\begin{table}[h!]
\caption{Generica tabella di probabilità congiunta.}
\label{tab_prob_cong}
\begin{center}
\begin{tabular}{C|CCCCCC}
\toprule
X \textbackslash Y & y_1 & y_2 & \dots & y_i & \dots & y_m\\
\midrule
x_1 & p(x_1,y_1) & p(x_1,y_2) & \dots & p(x_1,y_i) & \dots & p(x_1,y_m)\\
x_2 & p(x_2,y_1) & p(x_2,y_2) & \dots & p(x_2,y_i) & \dots & p(x_2,y_m)\\
\vdots & \vdots & \vdots & \vdots & \vdots & \vdots & \vdots\\
x_i & p(x_i,y_1) & p(x_i,y_2) & \dots & p(x_i,y_i) & \dots & p(x_i,y_m)\\
\vdots & \vdots & \vdots & \vdots & \vdots & \vdots & \vdots\\
x_n & p(x_n,y_1) & p(x_n,y_2) & \dots & p(x_n,y_i) & \dots & p(x_n,y_m)\\
\bottomrule
\end{tabular}
\end{center}
\end{table}%
In generale, possiamo dire che, dato uno spazio campionario discreto $\Omega$, è possibile associare ad ogni evento elementare $\omega_i$ dello spazio campionario  una coppia di numeri reali $(x, y)$, essendo $x = X(\omega)$ e $y = Y(\omega)$, il che ci conduce alla seguente definizione. 
\begin{defn}
La funzione che associa ad ogni coppia $(x, y)$ un livello di probabilità prende il nome di funzione di probabilità congiunta: 
\begin{equation}
P(x, y) = P(X = x, Y = y). 
\end{equation}
\end{defn}
Il termine ``congiunta'' deriva dal fatto che questa probabilità è legata al verificarsi di una coppia di valori, il primo associato alla variabile aleatoria $X$ ed il secondo alla variabile aleatoria $Y$. 
Nel caso di due sole variabili, si parla di distribuzione bivariata, mentre nel caso di più variabili si parla di distribuzione multivariata.

\subsection{Proprietà}

Una distribuzione di massa di probabilità congiunta bivariata deve soddisfare due proprietà:
\begin{enumerate}[label=(\alph*)]
\item $0 \leq p(x_i, y_j) \leq 1$;
\item la probabilità totale deve essere uguale a $1.0$. 
Tale proprietà può essere espressa nel modo seguente
\[
\sum_{i} \sum_{j} p(x_i, y_j) = 1.0.
\]
\end{enumerate}

\subsubsection{Eventi}

Si noti che dalla probabilità congiunta possiamo calcolare la probabilità di qualsiasi evento definito in base alle variabili aleatorie $X$ e $Y$. 
Per capire come questo possa essere fatto, consideriamo nuovamente l'esperimento discusso sopra. 
\begin{exmp}
Per la distribuzione di massa di probabilità congiunta riportata nella tabella~\ref{tab:tre_monete_distr_cong}, si trovi la probabilità dell'evento $X+Y \leq 1$.
\end{exmp}
\begin{solu}
Per trovare la probabilità richiesta dobbiamo semplicemente sommare le probabilità associate a tutte le coppie $(x,y)$ che soddisfano la conzione $X+Y \leq 1$. 
Ovvero,
\begin{align}
P_{XY}(X+Y \leq 1) = &P_{XY}(0, 0) + P_{XY}(1, 0)= 3/8.\notag
\end{align}
\end{solu}

\subsection{Funzioni di probabilità marginali}

Data la funzione di probabilità congiunta $p(x, y)$ è possibile pervenire alla costruzione della funzione di probabilità della singola variabile aleatoria, $X$ o $Y$: 
\[
p_X(x) = P(X = x) = \sum_y p(x,y)
\]
\[
p_Y(y) = P(Y = y) = \sum_x p(x,y)
\]
che prendono, rispettivamente, il nome di funzione di probabilità marginale di $X$ e funzione di probabilità marginale di $Y$.
Si noti che $P_X$ e $P_Y$ sono normalizzate:
\[
\sum_x P_X(x) = 1.0, \quad \sum_y P_Y(y) = 1.0.
\]

\begin{exmp}
Per l'esperimento casuale consistente nel lancio di tre monete equilibrate, si calcolino le probabilità marginali di $X$ e $Y$.
\end{exmp}
\begin{solu}
Nell'ultima colonna a destra e nell'ultima riga in basso della tabella~\ref{tab:tre_monete_distr_cong_2} sono riportate le distribuzioni di probabilità marginali di $X$ e $Y$.
$P_X$ si ottiene sommando su ciascuna riga fissata la colonna $j$, $P_X(X = j) = \sum_y p_{xy}(x = j, y)$. 
$P_Y$ si trova sommando su ciascuna colonna fissata la riga $i$, $P_Y (Y = i) = \sum_x p_{xy}(x, y = i)$. 
Si noti che:
\[
\sum_x P_X(x) = 1, \quad \sum_y P_Y(y) = 1.
\]
\begin{table}[h!]
\caption{Distribuzione di probabilità congiunta $p(x,y)$ per i risultati dell'esperimento consistente nel lancio di tre monete equilibrate e probabilità marginali $p(x)$ e $p(y)$.}
\label{tab:tre_monete_distr_cong_2}
\begin{center}
\begin{tabular}{c|cc|c}
\toprule
$x\textbackslash y$ & 0 & 1 & $p(x)$\\
\midrule
0 & 1/8 & 0   & 1/8 \\
1 & 2/8 & 1/8 & 3/8\\
2 & 1/8 & 2/8 & 3/8 \\
3 & 0   & 1/8 & 1/8 \\
\midrule
$p(y)$ & 4/8 & 4/8 & 1.0\\
\bottomrule
\end{tabular}
\end{center}
\label{default}
\end{table}%
\end{solu}

\subsection{Indipendenza stocastica}

Ora abbiamo tutti gli strumenti per potere dare una definizione statistica precisa al concetto di indipendenza. 
La definizione proposta sarà necessariamente coerente con la definizione di indipendenza che abbiamo usato fino ad ora. 
Ma, espressa in questi nuovi termini, potrà essere utilizzata in indagini probabilistiche e statistiche più complesse. 
Ricordiamo che gli eventi $A$ e $B$ si dicono indipendenti se 
$
P (A \cap B)\, = P(A) P(B)
$.
Diciamo quindi che $X$ e $Y$ sono indipendenti se qualsiasi evento definito da $X$ è indipendente da qualsiasi evento definito da $Y$. 
La definizione formale che garantisce che ciò accada è la seguente. 
\begin{defn}
Le variabili aleatorie distribuite congiuntamente $X$ e $Y$ sono indipendenti se la loro distribuzione congiunta è il prodotto delle distribuzioni marginali:
\begin{equation}
P(X, Y)\, = P_X(x)P_Y(y).
\end{equation}
\end{defn}

Nel caso discreto, dunque, l'indipendenza implica che la probabilità riportata in ciascuna cella della tabella di probabilità congiunta deve essere uguale al prodotto delle probabilità marginali di riga e di colonna:
\begin{equation}
p(x_i, y_i)\, = p_X(x_i) p_Y(y_i).\notag
\end{equation}


\begin{exmp}
\label{ex:three_coins_non_ind}
Per la situazione rappresentata nella tabella~\ref{tab:tre_monete_distr_cong_2} le variabili aleatorie $X$ e $Y$ sono indipendenti?
\end{exmp}
\begin{solu}
Nella tabella~\ref{tab:tre_monete_distr_cong_2} le variabili aleatorie $X$ e $Y$ non sono indipendenti: le probabilità congiunte non sono ricavabili dal prodotto delle marginali. 
Per esempio, nessuna delle probabilità marginali è uguale a $0$ per cui nessuno dei valori dentro la tabella (probabilità congiunte) che risulta essere uguale a $0$ può essere il prodotto delle probabilità marginali.
\end{solu}


\section*{Conclusioni}

La funzione di probabilità congiunta tiene simultaneamente conto del comportamento di due variabili aleatorie $X$ e $Y$ e di come esse si influenzano reciprocamente. 
In particolare, si osserva che se le due variabili non si influenzano, cioè se sono statisticamente indipendenti, allora la distribuzione di massa di probabilità congiunta si ottiene come prodotto delle funzioni di probabilità marginali di $X$ e $Y$: $P_{X, Y}(x, y) = P_X(x) P_Y(y)$.


%%-----------------------------------------------------------------------
%\section*{Problemi}
%\addcontentsline{toc}{section}{Problemi}
%
%
%\begin{prob}
%\label{ex:prob_cong_1}
%Una coppia di dadi è stata truccata in modo che al lancio di ogni dado la faccia 1 compaia con probabilità 1/3, mentre le altre facce restano tra loro equiprobabili. 
%Qual è la probabilità che lanciando la coppia di dadi la somma dei numeri sulle facce sia 7?
%\end{prob}
%
%%----------------------------------------------------------------------------
%\begin{prob}
%\label{ex:prob_cong_2}
%Una moneta viene lanciata tre volte. Sia $X$ il numero di ``teste'' nei primi due lanci e sia $Y$ il numero di ``teste'' sui tre lanci. Calcolare la distribuzione congiunta di $X$ e $Y$.
%\end{prob}
%
%%----------------------------------------------------------------------------
%\begin{prob}
%\label{ex:prob_cong_3}
%Un dado viene lanciato due volte. Sia $X$ la variabile aleatoria ``punteggio minimo'' e $Y$ la variabile aleatoria ``punteggio massimo''. Calcolare la distribuzione congiunta di $X$ e $Y$.
%\end{prob}

%%----------------------------------------------------------------------------
%\begin{prob}
%\label{ex:prob_cong_4}
%
%\end{prob}
%%----------------------------------------------------------------------------
%\begin{prob}
%\label{ex:prob_cong_5}
%
%\end{prob}
%%----------------------------------------------------------------------------
%\begin{prob}
%\label{ex:prob_cong_6}
%
%\end{prob}
%%----------------------------------------------------------------------------
%\begin{prob}
%\label{ex:prob_cong_7}
%
%\end{prob}


