\chapter{Terminologia}
\label{chapter:terminologia} 



\section{Metodi e procedure della psicologia}

Una teoria psicologica di un qualche aspetto del comportamento umano o della mente ha le seguenti proprietà:
\begin{enumerate}
\item
descrive le caratteristiche del comportamento in questione,
\item 
formula predizioni sulle caratteristiche future del comportamento,
\item 
è sostenuta da evidenze empiriche,
\item
deve essere falsificabile (ovvero, in linea di principio, deve potere fare delle predizioni su aspetti del fenomeno considerato che non sono ancora noti e che, se venissero indagati, potrebbero portare a rigettare la teoria, se si dimostrassero incompatibili con essa).
\end{enumerate}

L'analisi dei dati si riferisce al punto 3 indicato sopra e, nelle sue fasi distinte, ovvero 
\begin{itemize}
\item la misurazione, 
\item l'analisi descrittiva, 
\item l'inferenza causale, 
\end{itemize}
ha un ruolo centrale nello sviluppo delle teorie psicologiche.
Prima di affrontare il primo degli ambiti in cui abbiamo articolato l'analisi dei dati, ovvero quello della misurazione, prenderemo qui in esame la terminologia che viene utilizzata. 


\section{Variabili e costanti}

L'analisi dei dati inizia con l'individuazione delle unità portatrici di informazioni circa il fenomeno di interesse. 
Si dice \emph{popolazione} (o \emph{universo}) l'insieme $\Omega$ delle entità capaci di fornire informazioni sul fenomeno oggetto dell'indagine statistica. 
Possiamo dunque scrivere
$
\Omega = \{\omega_i\}_{i=1, \dots, n}= \{\omega_1, \omega_2, \dots, \omega_n\}
$
oppure
$
\Omega =  \{\omega_1, \omega_2, \dots \}
$
nel caso di popolazioni finite o infinite, rispettivamente. 
Gli elementi $\omega_i$ dell'insieme $\Omega$ sono detti \emph{unità statistiche}.
Un sottoinsieme della popolazione viene chiamato \emph{campione}.
Ciascuna unità statistica $\omega_i$ (abbreviata con u.s.) è portatrice dell'informazione che verrà rilevata mediante un'operazione di misurazione.

Definiamo \emph{variabile statistica} la proprietà (o grandezza) che è oggetto di studio nell'analisi dei dati. 
%La variabile statistica varia nella popolazione e, pertanto, viene anche detto \emph{carattere}. 
Una variabile è una proprietà di un fenomeno che può essere espressa in più valori sia numerici sia categoriali. 
Il termine \enquote{variabile} si contrappone al termine \enquote{costante} che descrive una proprietà invariante di tutte le unità statistiche.

Si dice \emph{modalità} ciascuna delle varianti con cui una variabile statistica può presentarsi.
Definiamo \emph{insieme delle modalità} di una variabile statistica l'insieme $M$ di tutte le possibili espressioni con cui la variabile può manifestarsi.
Le modalità osservate e facenti parte del campione si chiamano \emph{dati} (si veda la Tabella~\ref{tab:term_st_desc}). 


\begin{table}[h!]
\scriptsize
\begin{tabular}{p{2.0cm}|p{2.0cm}|p{2.0cm}|p{2.0cm}|p{2.0cm}}
\toprule
\textbf{Fenomeno studiato} & \textbf{Popolazione} & \textbf{Variabile} & \textbf{Modalità} & \textbf{Tipo}\\
\midrule
  Intelligenza & Tutti gli italiani & WAIS-IV & $112$, $92$, $121$, \dots & Quantitativo discreto \\
\midrule
Velocità di esecuzione nel compito Stroop & Bambini dai 6 agli 8 anni & Reciproco dei tempi di reazione & $1/2.36$ s, $1/4.72$ s, $1/3.81$ s, \dots & Quantitativo continuo\\
\midrule
Disturbo di personalità & Detenuti nelle carceri italiane & Assessment del disturbo di personalità tramite interviste cliniche strutturate & Cluster A, Cluster B, Cluster C proposti dal DSM-V & Qualitativo\\
  \bottomrule
\end{tabular}
\caption{Proprietà oggetto di studio, variabile e modalità.}
\label{tab:term_st_desc}
\end{table}


\section{Variabili indipendenti e variabili dipendenti}

È importante distinguere il concetto di \emph{variabile indipendente}, che descrive ciò che viene manipolato dallo sperimentatore o che è già presente nel campione, dalla \emph{variabile dipendente}, che descrive ciò che varia al variare della variabile indipendente e che viene misurato nel campione.

\begin{exmp}
Uno psicologo convoca 120 studenti universitari per un test di memoria. 
Prima di iniziare l'esperimento, a metà dei soggetti viene detto che si tratta di un compito particolarmente difficile; agli altri soggetti non viene data alcuna indicazione.
Lo psicologo misura il punteggio nella prova di memoria di ciascun soggetto. 
Si individuino la variabile indipendente e la variabile dipendente di questo esperimento.
\end{exmp}

\begin{solu}
La variabile indipendente è l'informazione sulla difficoltà della prova.
La variabile indipendente viene manipolata dallo sperimentatore assegnando i soggetti (di solito in maniera causale) o alla condizione (modalità) \enquote{informazione assegnata} o \enquote{informazione non data}.
La variabile dipendente è ciò che viene misurato nell'esperimento, ovvero il punteggio nella prova di memoria di ciascun soggetto.
\end{solu}


\section{La matrice dei dati}

Le realizzazioni delle variabili esaminate in una rilevazione statistica vengono  organizzate in una \emph{matrice dei dati}. 
Le colonne della matrice dei dati contengono gli insiemi dei dati individuali di ciascuna variabile statistica considerata. 
Ogni riga della matrice contiene tutte le informazioni relative alla stessa unità statistica. 
Una generica matrice dei dati ha l'aspetto seguente:
$$
D_{m,n} = 
 \begin{pmatrix}
  \omega_1 & a_{1}   & b_{1}   & \cdots & x_{1} & y_{1}\\
  \omega_2 & a_{2}   & b_{2}   & \cdots & x_{2} & y_{2}\\
  \vdots   & \vdots  & \vdots  & \ddots & \vdots & \vdots  \\
 \omega_n  & a_{n}   & b_{n}   & \cdots & x_{n} & y_{n}
 \end{pmatrix}
$$
dove, nel caso presente, la prima colonna contiene il nome delle unità statistiche, la seconda e la terza colonna si riferiscono a due mutabili statistiche (variabili categoriali; $A$ e $B$) e ne presentano le modalità osservate nel campione mentre le ultime due colonne si riferiscono a due variabili statistiche ($X$ e $Y$) e ne presentano le modalità osservate nel campione.
Generalmente, tra le unità statistiche $\omega_i$ non esiste un ordine progressivo; l'indice attribuito alle unità statistiche nella matrice dei dati si riferisce semplicemente alla riga che esse occupano.




%Consideriamo ora le $n$ unità del collettivo statistico $\Omega$. 
%Viene detto \emph{insieme dei dati individuali} di una mutabile statistica $A$ o di una variabile statistica $X$ l'insieme costituito dalle  loro $n$ determinazioni. 
%Gli insiemi dei dati individuali contengono tutte le informazioni circa i caratteri rilevati. Le  elaborazioni statistiche volte a descrivere il fenomeno di interesse hanno come oggetto l'insieme dei dati individuali.


%\section{Classificazione delle variabili statistiche}
%
%%I dati utilizzati in un'indagine statistica sono il risultato di un processo di misurazione. 
%%Di conseguenza, essi rappresentano le proprietà del fenomeno empirico oggetto di studio con gradi diversi di approssimazione, a seconda della validità e dell'attendibilità dello strumento di misurazione con cui sono stati ottenuti. 
%
%Le variabili statistiche possono essere suddivise in due categorie: variabili qualitative e variabili quantitative.
%Le \emph{variabili qualitative} descrivono una qualità piuttosto che una quantità numerica. 
%Per esempio, consideriamo la domanda ``Qual è il tuo cibo preferito?''. 
%Risposte a tale domanda possono essere: mirtilli, cioccolato, pasta, pizza e mango. 
%Questi dati non sono numerici; tuttavia è possibile assegnare un numero a ciascuna categoria di risposta (1 = mirtilli, 2 = cioccolato, ecc.), ma in questo caso i numeri sarebbero semplicemente delle etichette e su tali numeri non avrebbe alcun senso eseguire qualche operazione aritmetica. 
%Per esempio, non avrebbe senso sommare tali numeri. 
%In contrasto con le variabili qualitative, le \emph{variabili quantitative} utilizzano i numeri per codificare l'intensità del fenomeno oggetto di interesse. 
%Uno dei vantaggi dei dati quantitativi è la possibilità di riassumere le nostre osservazioni mediante degli indici numerici che facilitano la comprensione e l'utilizzo dei dati. 
%Tali indici sono chiamati \emph{statistiche}.
%Esempi sono: la media, la mediana, la varianza, la differenza inter-quartile, \dots

%È possibile classificare le variabili statistiche nel modo seguente.
%\begin{description}
%  \item[Le variabili qualitative] possono essere suddivise in
%  \begin{itemize}
%    \item{\emph{caratteri qualitativi sconnessi}: hanno per modalità denominazioni qualitative tra le quali non è possibile stabilire alcuna relazione d'ordine naturale;}
%    \item{\emph{caratteri qualitativi ordinali}: hanno per modalità denominazioni qualitative che presentano presentano una relazione d'ordine naturale.}
%    \end{itemize}
%  \item[Le variabili quantitative] possono essere suddivise in
%  \begin{itemize}
%  \item{\emph{caratteri quantitativi discreti}: sono caratterizzati dal fatto che, fissata una modalità, esiste un intervallo all'interno del quale non esiste alcun valore che costituisce una modalità}
%  \item{\emph{caratteri quantitativi continui}: per qualunque intervallo tra due modalità tra loro esistono comunque infiniti valori che sono altrettante modalità.}
%\end{itemize}
%\end{description}




