\chapter{Probabilità condizionata}
\label{chapter:prob_cond} 

%\section*{Obiettivi di apprendimento}
%
%Lo studio di questo capitolo dovrebbe insegnare allo studente come:
%\begin{itemize}
%\item Spiegare il significato dei termini probabilità condizionata e indipendenza.
%\item Descrivere la legge della probabilità composta.
%\item Determinare se due eventi sono indipendenti.
%\end{itemize}
%
%\section*{Motivazione}

L'attribuzione di una probabilità ad un evento è sempre condizionata dalle conoscenze che abbiamo a disposizione.  
Per un determinato stato di conoscenze, attribuiamo ad un dato evento una certa probabilità di verificarsi; ma se il nostro stato di conoscenze cambia, allora cambierà anche la probabilità che attribuiamo all'evento in questione. 
Per esempio, posiamo chiederci quale sia probabilità che Mario Rossi superi l'esame di Psicometria nel primo appello del presente anno accademico.  
In assenza di altre informazioni, la migliore stima di tale probabilità è data dalla proporzione di studenti che hanno superato l'esame di Psicometria nel corrispondente appello dei passati anni accademici. 
Ma se sappiamo che Mario Rossi è particolarmente portato per le materie quantitative, ha un'ottima preparazione di base e ha studiato molto, allora la probabilità sarà sicuramente più alta.


\section{Probabilità condizionata su altri eventi}
\label{sec:prob_cond}

La probabilità condizionata è una componente essenziale del ragionamento scientifico dato che chiarisce come sia possibile incorporare le evidenze disponibili, in maniera logica e coerente, nella nostra conoscenza del mondo.
Infatti, si può pensare che tutte le probabilità siano probabilità condizionate, anche se l'evento condizionante non è sempre esplicitamente menzionato. 
Consideriamo il seguente problema.

\begin{exmp}
Lo screening per la diagnosi precoce del tumore mammario si avvale di test che sono accurati al 90\%, nel senso che il 90\% delle donne con cancro e il 90\% delle donne senza cancro saranno classificate correttamente. 
Supponiamo che l'1\% delle donne sottoposte allo screening abbia effettivamente il cancro al seno. 
Ci chiediamo: qual è la probabilità che una donna scelta casualmente abbia una mammografia positiva e, se ce l'ha, qual è la probabilità che abbia davvero il cancro?
\end{exmp}
\label{ex:cancro_seno}
\begin{solu}
Per risolvere questo problema, supponiamo che il test in questione venga somministrato ad un grande campione di donne, diciamo a \num{1000} donne.
Di queste \num{1000} donne, 10 (ovvero, l'1\%) hanno il cancro al seno.
Per queste 10 donne, il test darà un risultato positivo in 9 casi (ovvero, nel 90\% dei casi). 
Per le rimanenti 990 donne che non hanno il cancro al seno, il test darà un risultato positivo in 99 casi (se la probabilità di un vero positivo è del 90\%, la probabilità di un falso positivo è del 10\%).
Questa situazione è rappresentata nella figura~\ref{fig:mammografia}.
Mettendo insieme questi due risultati, vediamo che il test dà un risultato positivo per 9 donne che hanno effettivamente il cancro al seno e per 99 donne che non ce l'hanno, per un totale di 108 risultati positivi.
Dunque, la probabilità di ottenere un risultato positivo al test è $\frac{108}{1000}$ = 11\%.
Ma delle 108 donne che hanno ottenuto un risultato positivo al test, solo 9 hanno il cancro al seno.
Dunque, la probabilità di avere il cancro, dato un risultato positivo al test, è  pari a $\frac{9}{108}$ = 8\%.

\begin{figure}
\centering
\begin{forest}
  [1000
    [10 cancro, label=left:Stato del mondo
      [$9^+$, label=left:Risultati della mammografia]
      [$1^-$]
    ]
    [990 non cancro
      [$99^+$]
      [$891^-$]
    ]
  ]
\end{forest}
\caption{Rappresentazione ad albero che riporta le frequenze attese dei risultati di una mammografia in un campione di \num{1000} donne.}
\label{fig:mammografia}
\end{figure}

\end{solu}

Nell'esercizio~\ref{ex:cancro_seno} la probabilità dell'evento \enquote{ottenere un risultato positivo al test} è una probabilità non condizionata, mentre la probabilità dell'evento \enquote{avere il cancro al seno, dato che il test ha dato un risultato positivo} è una probabilità condizionata.
In termini generali, la probabilità condizionata $P(A \mid B)$ rappresenta la probabilità che si verifichi l'evento $A$ sapendo che si è verificato l'evento $B$; oppure: la probabilità di $A$ in una prova valida solo se si verifica anche $B$. 
Ciò ci conduce alla seguente definizione. 

\begin{defn}
Dato un qualsiasi evento $A$, si chiama \emph{probabilità condizionata} di $A$ dato $B$ il numero
\begin{equation}
P(A \mid B) = \frac{P(A \cap B)}{P(B)}, \quad \text{con}\, P(B) > 0,
\label{eq:probcond}
\end{equation}
dove $P(A\cap B)$ è la probabilità congiunta dei due eventi, ovvero la probabilità che si verifichino entrambi.
\end{defn}

In alcuni casi può essere conveniente leggere al contrario la formula~\eqref{eq:probcond} e utilizzarla per calcolare la probabilità dell'intersezione di due eventi. Per esempio se conosciamo la probabilità dell'evento $B$ e la probabilità condizionata di $A$ su $B$, otteniamo
\begin{equation}
P(A \cap B) = P(B)P(A \mid B),
\label{eq:probcondinv}
\end{equation}
mentre se conosciamo la probabilità dell'evento $A$ e la probabilità condizionata di $B$
su $A$, otteniamo
$P(A \cap B) = P(A)P(B \mid A)$.

\begin{exmp}
Da un mazzo di 52 carte (13 carte per ciascuno dei 4 semi) ne viene estratta 1 in modo casuale. 
Qual è la probabilità che esca una figura di cuori? Sapendo che la carta estratta ha il seme di cuori, qual è la probabilità che il valore numerico della carta sia 7, 8 o 9? 
\end{exmp}

\begin{solu}
Ci sono 13 carte di cuori, dunque la risposta alla prima domanda è 1/4. 
Per rispondere alla seconda domanda consideriamo solo le 13 carte di cuori; la probabilità cercata è dunque 3/13.
\end{solu}


\subsection{La fallacia del pubblico ministero}

Un errore comune che si commette è quello di credere che $P(A \mid B)$ sia uguale a $P(B \mid A)$. 
Tale fallacia ha particolare risalto in ambito forense tanto che è conosciuta con il nome di \enquote{fallacia del procuratore} (\emph{prosecutor's fallacy}).
%La fallacia del procuratore è una fallacia del ragionamento statistico, utilizzato dalla procura, per sostenere la colpevolezza di un imputato in un processo penale.
In essa, una piccola probabilità dell'evidenza, data l'innocenza, viene erroneamente interpretata come la probabilità dell'innocenza, data l'evidenza.

Consideriamo il caso di un esame del DNA.
Un esperto forense potrebbe affermare, ad esempio, che \enquote{se l'imputato è innocente, c'è solo una possibilità su un miliardo che vi sia una corrispondenza tra il suo DNA e il DNA trovato sulla scena del crimine}. 
Ma talvolta questa probabilità è erroneamente interpretata come avesse il seguente significato: \enquote{date le prove del DNA, c'è solo una possibilità su un miliardo che l'imputato sia innocente}.

Le considerazioni precedenti risultano più chiare se facciamo nuovamente riferimento al problema~\ref{ex:cancro_seno}.
In tale esercizio abbiamo visto come la probabilità di cancro dato un risultato positivo al test sia uguale a 0.08. 
Tale probabilità è molto diversa dalla probabilità di un risultato positivo al test data la presenza del cancro.
Infatti, questa seconda probabilità è uguale a 0.90 ed è descritta nel problema come una delle caratteristiche del test in questione.


\section{Legge della probabilità composta}

Il teorema della probabilità composta deriva dal concetto di probabilità condizionata per cui la probabilità che si verifichino due eventi $A_i$ e $A_j$ è pari alla probabilità di uno dei due eventi moltiplicato con la probabilità dell'altro evento condizionato al verificarsi del primo. 
\begin{teorema}
L'equazione~\eqref{eq:probcondinv} si estende al caso di $n$ eventi $A_1, \dots, A_n$ nella forma seguente:
\begin{equation} 
\begin{split}
P(A_1 \cap A_2 \cap \dots\cap A_n) = {}& P(A_1)P(A_2 \mid A_1)P(A_3 \mid A_1 \cap A_2) \dots\\
 & P(A_n \mid A_1 \cap A_2 \cap \dots \cap A_{n-1})
\end{split}
\label{eq:probcomposte}
\end{equation}
che esprime in forma generale la legge della probabilità composta.
\end{teorema}

\begin{exmp}
Da un'urna contenente 6 palline bianche e 4 nere si estrae una pallina per volta, senza reintrodurla nell'urna. 
Indichiamo con $B_i$ l'evento: ``esce una pallina bianca alla $i$-esima estrazione'' e con $N_i$ l'estrazione di una pallina nera. 
L'evento: ``escono due palline bianche nelle prime due estrazioni'' è rappresentato dalla intersezione $\{B_1 \cap  B_2\}$ e la sua probabilità vale, per la~\eqref{eq:probcondinv}, 
$$
P(B_1 \cap B_2) = P(B_1)P(B_2 \mid B_1).
$$ 
$P(B_1)$ vale 6/10, perché nella prima estrazione $\Omega$ è costituito da 10 elementi: 6 palline bianche e 4 nere. 
La probabilità condizionata $P(B_2 \mid B_1)$ vale 5/9, perché nella seconda estrazione, se è verificato l'evento $B_1$, lo spazio campionario consiste di 5 palline bianche e 4 nere. 
Si ricava pertanto: $$P(B_1 \cap B_2) = \frac{6}{10} \cdot \frac{5}{9} = \frac{1}{3}.$$ 
In modo analogo si ha che
$$
P(N_1 \cap N_2) = P(N_1)P(N_2 \mid N_1) = \frac{4}{10} \cdot \frac{3}{9} = \frac{4}{30}.
$$

Se l'esperimento consiste nell'estrazione successiva di 3 palline, la probabilità che
queste siano tutte bianche vale, per la~\eqref{eq:probcomposte}:
$$
P(B_1 \cap B_2 \cap B_3)=P(B_1)P(B_2 \mid B_1)P(B_3 \mid B_1 \cap B_2),
$$
dove la probabilità $P(B_3 \mid B_1 \cap B_2)$ si calcola supponendo che si sia verificato l'evento condizionante $\{B_1 \cap B_2\}$. 
Lo spazio campionario per questa probabilità condizionata è costituito da 4 palline bianche e 4 nere, per cui $P(B_3 \mid B_1 \cap B_2) = 1/2$ e quindi: 
$$
P (B_1 \cap B_2 \cap B_3) = \frac{6}{10}\cdot\frac{5}{9} \cdot\frac{4}{8}  = \frac{1}{6}. 
$$

La probabilità dell'estrazione di tre palline nere è invece:
\begin{align}
P(N_1 \cap N_2 \cap N_3) &= P(N_1)P(N_2 \mid N_1)P(N_3 \mid N_1 \cap N_2)\notag\\ 
&= \frac{4}{10} \cdot \frac{3}{9} \cdot \frac{2}{8} = \frac{1}{30}.\notag
\end{align}

\end{exmp}


\section{L'indipendendenza stocastica}

Un concetto molto importante per le applicazioni statistiche della probabilità è quello dell'indipendenza stocastica. La definizione~\eqref{eq:probcond} esprime il concetto intuitivo di indipendenza di un evento da un altro, nel senso che il verificarsi di $A$ non influisce sulla probabilità del verificarsi di $B$, ovvero non la condiziona. Infatti, per la definizione~\eqref{eq:probcond} di probabilità condizionata, si ha che, se $A$ e $B$ sono due eventi indipendenti, risulta:   
\begin{equation}
P(A \mid B) = \frac{P(A)P(B)}{P(B)} = P(A).\notag
\end{equation}
Possiamo dunque dire che due eventi $A$ e $B$ sono indipendenti se
\begin{equation}
\begin{split}
P(A \mid B) &= P(A), \\
P(B \mid A) &= P(B).
\end{split}
\end{equation}

\begin{exmp}
Nel lancio di due dadi non truccati, si considerino gli eventi: $A=\{$esce un $1$ o un $2$ nel primo lancio$\}$ e $B=\{$il punteggio totale è $8\}$. 
Gli eventi $A$ e $B$ sono indipendenti?
\end{exmp}

\begin{solu}
Rappresentiamo qui sotto lo spazio campionario dell'esperimento casuale.
\begin{center}
{\large
\noindent
\epsdice{1}\epsdice{1}\, \epsdice{1}\epsdice{2}\, \epsdice{1}\epsdice{3}\, 
\epsdice{1}\epsdice{4}\, \epsdice{1}\epsdice{5}\, \epsdice{1}\epsdice{6}\\
\epsdice{2}\epsdice{1}\, \epsdice{2}\epsdice{2}\, \epsdice{2}\epsdice{3}\, 
\epsdice{2}\epsdice{4}\, \epsdice{2}\epsdice{5}\, \epsdice[black]{2}\epsdice[black]{6}\\
\epsdice{3}\epsdice{1}\, \epsdice{3}\epsdice{2}\, \epsdice{3}\epsdice{3}\, 
\epsdice{3}\epsdice{4}\, \epsdice[black]{3}\epsdice[black]{5}\, \epsdice{3}\epsdice{6}\\
\epsdice{4}\epsdice{1}\, \epsdice{4}\epsdice{2}\, \epsdice{4}\epsdice{3}\, 
\epsdice[black]{4}\epsdice[black]{4}\, \epsdice{4}\epsdice{5}\, \epsdice{4}\epsdice{6}\\
\epsdice{5}\epsdice{1}\, \epsdice{5}\epsdice{2}\, \epsdice[black]{5}\epsdice[black]{3}\, 
\epsdice{5}\epsdice{4}\, \epsdice{5}\epsdice{5}\, \epsdice{5}\epsdice{6}\\
\epsdice{6}\epsdice{1}\, \epsdice[black]{6}\epsdice[black]{2}\, \epsdice{6}\epsdice{3}\, 
\epsdice{6}\epsdice{4}\, \epsdice{6}\epsdice{5}\, \epsdice{6}\epsdice{6}\\
}
\end{center}
Gli eventi $A$ e $B$ non sono statisticamente indipendenti. 
Infatti, le loro probabilità valgono $P(A) = 12/36$ e $P(B) = 5/36$ e la probabilità della loro intersezione è 
$P(A \cap B) = 1/36 = 3/108 \neq P(A)P(B) = 5/108$.
\end{solu}

\begin{oss}
Si noti che il concetto di indipendenza è del tutto differente da quello di incompatibilità. 
Due eventi $A$ e $B$ incompatibili (per i quali si ha $A \cap B = \emptyset$) sono statisticamente dipendenti, poiché il verificarsi dell'uno esclude il verificarsi dell'altro: $P(A \cap B)=0  \neq P(A)P(B)$. 
\end{oss}

\begin{oss}
Se due eventi con probabilità non nulla sono statisticamente indipendenti, la legge delle probabilità totali espressa dalla~\eqref{eq:probunione} si modifica nella relazione seguente:
\begin{equation}
P(A \cup B) = P(A) + P(B) - P(A)P(B).\notag
\end{equation}
\end{oss}

\section*{Conclusioni}

La probabilità condizionata è importante perché ci fornisce uno strumento per precisare il concetto di indipendenza statistica.
Una delle più importanti domande delle analisi statistiche è infatti quella che si chiede se due variabili siano o meno associate.
In questo capitolo abbiamo discusso il concetto di indipendenza (come contrapposto al concetto di associazione); nel capitolo~\ref{chapter:descriptive_stats} abbiamo descritto poi uno dei modi possibili che ci consentono di quantificare l'associazione tra due variabili.
In seguito vedremo come sia possibile fare inferenza sull'associazione tra variabili -- ovvero, come stabilire il livello di fiducia nel verificarsi dell'evento esaminato nel campione in un contesto più ampio, cioè quello della popolazione.
 


%%%-----------------------------------------------------------------------------
%\section*{Problemi}
%
%\addcontentsline{toc}{section}{Problemi}
%
%%--------------------------------------------------------------------
%\begin{prob}
%\label{ex:dobrow_2_2}
%Siano $P(A) = P(B) = 0.3$ e $P(A \mid B) = 0.5$. Si trovi $P(A \cup B)$.
%\end{prob}
%
%%--------------------------------------------------------------------
%\begin{prob}
%\label{ex:dobrow_2_3}
%Siano $P(A) = P(B) = p_1$ e $P(A \cup B) = p_2$. Si trovi $P(A \mid B)$.
%\end{prob}
%
%%--------------------------------------------------------------------
%\begin{prob}
%\label{ex:dobrow_2_9}
%Nel poker, si chiama \emph{flush} un insieme di 5 carte tutte dello stesso seme (non necessariamente consecutive). Si usi la probabilità condizionata per calcolare la probabilità di ricevere 5 carte che costituiscono un \emph{flush}.
%\end{prob}
%
%%--------------------------------------------------------------------
%\begin{prob}
%\label{ex:ripmat_1}
%Trovare la probabilità, estraendo una carta da un mazzo di 40, e rimettendo la carta nel mazzo prima della seconda estrazione, di estrarre due volte l'asso di bastoni.
%\end{prob}
%
%%--------------------------------------------------------------------
%\begin{prob}
%\label{ex:ripmat_2}
%Abbiamo un'urna con 10 palline bianche, 20 rosse e 30 nere: trovare la probabilità di estrarre successivamente due palline bianche senza rimettere la prima pallina estratta nell'urna.
%\end{prob}
%
%%--------------------------------------------------------------------
%\begin{prob}
%\label{ex:ripmat_3}
%Nel gioco della roulette vi sono 36 numeri più lo zero. Trovare la probabilità alla roulette di fare due volte ``en plein'' su due giocate successive puntando sul numero 3.
%\end{prob}
%
%%--------------------------------------------------------------------
%\begin{prob}
%\label{ex:ripmat_4}
%Trovare la probabilità, nel lancio di una moneta, di fare "testa" 4 volte consecutive.
%\end{prob}
%
%%--------------------------------------------------------------------
%\begin{prob}
%\label{ex:shay_443}
%Vengono lanciati due dadi. Qual è la probabilità che la somma dei punti prodotti dai due dadi sia due o tre, dato che almeno un dado mostra un 1?
%\end{prob}
%
%%--------------------------------------------------------------------
%\begin{prob}
%\label{ex:shay_445}
%Da tutte le famiglie con tre figli, selezioniamo una famiglia a caso. Qual è la probabilità che i bambini siano tutti maschi, se sappiamo che a) il primo è un ragazzo e b) almeno uno è un maschio? (Supponiamo che la probabilità di un bambino maschio sia sempre 1/2).
%\end{prob}
%
%%--------------------------------------------------------------------
%\begin{prob}
%\label{ex:shay_451}
%Supponiamo di avere due urne, con la prima contenente due palline bianche e sei nere e la seconda contenente due palline bianche e due nere. Selezioniamo un'urna a caso e poi estraiamo a caso una pallino dall'urna scelta. Qual è la probabilità di estrarre una pallina bianca?
%\end{prob}
%
%%--------------------------------------------------------------------
%\begin{prob}
%\label{ex:shay_452}
%Da un mazzo di 52 carte, tre sono estratte senza rimessa. Qual è la probabilità dell'evento $E$ di ottenere due assi e un re in qualsiasi ordine?
%\end{prob}
%
%%--------------------------------------------------------------------
%\begin{prob}
%\label{ex:condprobstackexchange_1}
%Un dado bilanciato viene lanciato. Sapendo che è uscito un numero pari, qual è la probabilità di osservare l'esito ``2''?
%\end{prob}
