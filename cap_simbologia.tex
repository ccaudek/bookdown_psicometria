\subsection*{Simbologia di base}

Per una scrittura più sintetica possono essere utilizzati alcuni simboli matematici.

\begin{itemize}
\item L'operatore logico booleano $\land$ significa ``e'' (congiunzione forte) mentre il  connettivo di disgiunzione $\lor$ significa ``o'' (oppure) (congiunzione debole).
\item Il quantificatore esistenziale $\exists$ vuol dire ``esiste almeno un'' e indica 
l'esistenza di almeno una istanza del concetto/oggetto indicato. Il quantificatore 
esistenziale di unicità $\exists!$ (``esiste soltanto un'') indica l'esistenza di 
esattamente una istanza del concetto/oggetto indicato. Il quantificatore esistenziale 
$\nexists$ nega l'esistenza del concetto/oggetto indicato.
\item Il quantificatore universale $\forall$ vuol dire ``per ogni.''
\item L'implicazione logica ``$\Rightarrow$'' significa ``implica'' (se \dots allora). $P \Rightarrow Q$ vuol dire che $P$ è condizione sufficiente per la verità di $Q$ e che $Q$ è condizione necessaria per la verità di $P$. 
\item L'equivalenza matematica ``$\iff$'' significa ``se e solo se'' e indica una condizione necessaria e sufficiente, o corrispondenza biunivoca.
\item Il simbolo $\vert$ si legge ``tale che.''
\item Il simbolo $\triangleq$ (o $:=$) si legge ``uguale per definizione.'' 
\item Il simbolo $\Delta$ indica la differenza fra due valori della variabile scritta a destra del simbolo. 
% \item 
%Il simbolo $\to$ vuol dire ``tende a'' e viene usato, per esempio, sotto al simbolo di ``limite.''
\item Il simbolo $\propto$ si legge ``proporzionale a.''
\item Il simbolo $\approx$ si legge ``circa.''
\item Il simbolo $\in$ della teoria degli insiemi vuol dire ``appartiene'' e indica l'appartenenza di un elemento ad un insieme. Il simbolo $\notin$ vuol dire ``non appartiene.''
\item Il simbolo $\subseteq$ si legge ``è un sottoinsieme di'' (può coincidere con l'insieme stesso). Il simbolo $\subset$ si legge ``è un sottoinsieme proprio di.'' 
\item Il simbolo $\#$ indica la cardinalità di un insieme.
\item Il simbolo $\cap$ indica l'intersezione di due insiemi. Il simbolo $\cup$ indica l'unione di due insiemi.
\item Il simbolo $\emptyset$ indica l'insieme vuoto o evento impossibile.
\end{itemize}
